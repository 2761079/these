% !TEX root = manuscrit.tex


\newcommand{\comment}[1]{\marginpar{\small{#1}}}
%\newcommand{\comment}[1]{\marginpar{\phantom{\small{#1}}}}

\newcommand{\arena}{\ensuremath{\mathcal{A}}}
\newcommand{\Win}{\ensuremath{\mathit{Win}}}
\newcommand{\Reach}{\ensuremath{\mathit{Reach}}}
\newcommand{\Outcome}{\ensuremath{\mathit{Outcome}}}
\newcommand{\N}{\ensuremath{\mathbb{N}}}
\newcommand{\Part}{\ensuremath{\mathcal{P}}}
\newcommand{\view}{\ensuremath{\mathit{view}}}
\newcommand{\Views}{\ensuremath{\mathcal{V}}}
\newcommand{\obs}{\ensuremath{\mathit{obs}}}
\newcommand{\equivclass}[1]{[#1]_\Oequiv}
\newcommand{\Aequivclass}[1]{[#1]_{\Oequiv_{\Past}}}
\newcommand{\Act}{\ensuremath{\mathbb{M}}}
\newcommand{\Rep}{\ensuremath{\mathit{Rep}}}
\newcommand{\Config}{\ensuremath{\mathcal{C}}}
\newcommand{\Agather}{\ensuremath{\arena_{\textrm{gather}}}}
\newcommand{\rep}{\ensuremath{\mathit{rep}}}
\newcommand{\Sum}[3]{\sum\limits_{#1}^{#2} #3}
\newcommand{\Next}{\ensuremath{\mathit{Next}}}
\newcommand{\eff}{\oplus}
%front back ....
\newcommand{\clockwise}{\ensuremath{\textit{Front}}}
\newcommand{\Front}{\clockwise}
\newcommand{\counterclockwise}{\ensuremath{\textit{Back}}}
\newcommand{\Back}{\counterclockwise}
\newcommand{\still}{\ensuremath{\textit{Idle}}}
\newcommand{\Idle}{\still}
\newcommand{\cclockwise}{\ensuremath{\textit{Clockwise}}}
\newcommand{\cFront}{\cclockwise}
\newcommand{\ccounterclockwise}{\ensuremath{\textit{Anti-clockwise}}}
\newcommand{\cBack}{\ccounterclockwise}
\newcommand{\wait}{\ensuremath{\bot}}
\newcommand{\?}{\ensuremath{\textit{Doubt}}}
\newcommand{\disoriented}{\?}
\newcommand{\Doubt}{\?}
\newcommand{\Actions}{\ensuremath{\Delta}}
\newcommand{\Past}{\ensuremath{S}}
\newcommand{\PastK}{\ensuremath{S^{k}}}
\newcommand{\PosTower}{\ensuremath{\mathit{PosTower}}}
\newcommand{\f}{\ensuremath{\mathfrak{f}}}
\newcommand{\Look}{\ensuremath{\mathit{Look}}}
\newcommand{\Move}{\ensuremath{\mathit{Move}}}
\newcommand{\AConfig}{\Config_{\Past}}
\newcommand{\m}{\ensuremath{\mathit{mem}}}
\newcommand{\maxmult}{\ensuremath{\mathit{maximal\_multiplicity}}\xspace}
\newcommand{\observe}{\ensuremath{\mathit{observe\_neighbors}}\xspace}
\newcommand{\Dslash}{\texttt{\symbol{92}\symbol{92}}}
\newcommand{\plusN}{\ensuremath{+_n}}
\newcommand{\plusK}{\ensuremath{+_k}}
\newcommand{\moinsN}{\ensuremath{-_n}}
\newcommand{\moinsK}{\ensuremath{-_k}}
\newcommand{\sym}{\ensuremath{\textit{sym}}}
\newcommand{\suc}{\ensuremath{\textit{succ}}}
\newcommand{\st}{\ensuremath{\textit{ such that }}}
\newcommand{\ie}{\ensuremath{\textit{i.e.}}, }
\newcommand{\mem}{\ensuremath{s}}
\newcommand{\ObsClass}{\ensuremath{\Obs/\!\!\!\Oequiv}}
%decision function
\newcommand{\dec}{\ensuremath{\partial}}





\newcommand{\Oequiv}{\ensuremath{\equiv}}
\newcommand{\Cequiv}{\ensuremath{\approx}}


\newcommand{\Rob}{\ensuremath{\textit{Rob}}}
\newcommand{\Pos}{\ensuremath{\textit{Pos}}}
\newcommand{\Obs}{\ensuremath{\textit{Obs}}}
\newcommand{\Sched}{\ensuremath{\textit{Sched}}}
\newcommand{\LC}{\ensuremath{\textit{LC}}}
\newcommand{\RLC}{\ensuremath{\textit{RLC}}}
%\newcommand{\O}{\ensuremath{\mathcal{O}}} \o \O already defined






\newcommand{\G}{\Box}
\newcommand{\F}{\Diamond}
\newcommand{\U}{\mathsf{U}}
\newcommand{\X}{\mathsf{X}}
\newcommand{\T}{\mathcal{T}}
\newcommand{\true}{\textit{true}}
\newcommand{\false}{\textit{false}}



\newtheorem{theorem}{Theorem}
\newtheorem{lemma}[theorem]{Lemma}
\newtheorem{proposition}[theorem]{Proposition}
\newtheorem{definition}{Definition}
\newtheorem{example}{Example}


\usepackage[]{algorithm2e}
\usepackage{wrapfig}

\usepackage{etoolbox}
\newcommand*{\lastlevel}{section}
\pretocmd{\section}{\def\lastlevel{section}}{}{}
\pretocmd{\subsection}{\def\lastlevel{subsection}}{}{}
\pretocmd{\subsubsection}{\def\lastlevel{subsubsection}}{}{}

\makeatletter
\newcommand{\l@todo}[2]{%
  \begingroup
    \let\@dotsep\@M
    \csname l@#2\endcsname{#1}{}%
  \endgroup
}

\newcommand{\todo}[1]{
  \ifdefstring{\lastlevel}{section}{%
    \addtocontents{toc}{\protect\vspace{-1.0em plus -1pt}}%
  }{}%
  \addtocontents{toc}{%
    \protect\contentsline{todo}{\protect\numberline{}\normalfont\itshape
  \textcolor{red}{TODO #1}}{\lastlevel}}%
   \textcolor{red}{{TODO #1}}
}
\makeatother
